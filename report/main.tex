\documentclass[12pt,twoside]{article}
\usepackage{jmlda}
\title
%    [Нелинейное ранжирование результатов разведочного информационного поиска] % Краткое название; не нужно, если полное название влезает в~колонтитул
{Quality prediction of proteins models with spherical convolutions on three-dimensional graphs.}
\author
%    [Мамонов~К.\,Р.] % список авторов для колонтитула; не нужен, если основной список влезает в колонтитул
{Nikita Pavlichenko, Sergei Grudinin, Ilia Igashov.} % основной список авторов, выводимый в оглавление
%[Мамонов~К.\,Р.$^1$, Воронцов~К.\,В.$^1$, Еремеев~М.\,А.$^1$] % список авторов, выводимый в заголовок; не нужен, если он не отличается от основного
%\thanks
%    {Работа выполнена при финансовой поддержке РФФИ, проект \No\,00-00-00000.
%   Научный руководитель:  Стрижов~В.\,В.
%{Задачу поставил:  Воронцов~К.\,В.
%	 Консультант:  Еремеев~М.\,А.}
\email
{pavlichenko.nv@phystech.ru, sergei.grudinin@inria.fr, igashov.i@yandex.ru}
\organization
{$^1$ Moscow Institute of Physics and Technolgy, Moscow, Russia}%; $^2$Организация}
\abstract
{Convolutional neural networks have become very popular in recent years, and, in particular, have found widespread application in computer vision. Recently, active work has also begun on graph convolutional networks. In general, the graphs, unlike the pictures, are irregular structures, and in many tasks of learning on graphs sample objects also do not have unified topology. Therefore, the existing operations of convolution on the graphs are very much simplified, and the task of pulling on the graphs remain open in general. The purpose of this work is to research new operations of convolution on three-dimensional graphs within the framework of solving the problem of quality estimation of three-dimensional models of proteins (the problem of regression on the graph nodes).
	
	\bigskip
	\textbf{Key words}: \emph {graph convolutional networks, spherical convlutions, three-dimensional graphs learning}.}
\titleEng
{Quality prediction of proteins models with spherical convolutions on three-dimensional graphs}
\authorEng
{Author~F.\,S.$^1$, CoAuthor~F.\,S.$^2$, Name~F.\,S.$^2$}
\organizationEng
{$^1$Moscow Institute of Physics and Technolgy, Moscow, Russia;}
\abstractEng
{This document is an example of paper prepared with \LaTeXe\
	typesetting system and style file \texttt{jmlda.sty}.
	
	\bigskip
	\textbf{Keywords}: \emph{keyword, keyword, more keywords}.}
\begin{document}
	
	\maketitle
	
	\section{Introduction}
	Protein molecules are an important part of any biological form. They determine cellular 
	functions and behavior of various biological and chemical structures. It makes the discovery 
	and prediction of proteins structure one of the most important points of medical, chemical 
	and genetic science researches.

	Molecules of proteins consist of smaller molecules called amino acids. These amino acids
	form a chain that is folded and placed in space. Thus, protein functions are determined
	by their positions in a 3D space. So, having this chain of amino acids we need to identify
	how they are located. There are ways to do this experimentally, but it can be time-consuming,
	expensive and not always possible. To solve these disadvantages, computational algorithms \cite{Arnold2005}\cite{Lundstroem2008}\cite{Xu2019}
	were developed that generate different chain foldings. The problem is that no algorithm is the 
	best one. Some of proteins are better modeled by one algorithm, others by others. Therefore, we 
	are facing the problem of quality assessment (QA) of these protein models.

	This problem has recently got attention from the machine learning community. Various artificial
	intelligence methods were applied such as neural networks \cite{Wallner2003} and support vector machines \cite{Ray2012}\cite{Uziela2016}.
	More recent approaches mostly include deep learning methods \cite{Hurtado2018}\cite{Derevyanko2018}\cite{Pages2019}\cite{Conover2019}. The newest approach is to use graph
	machine learning methods such as Graph Convolutional Networks (GCN) \cite{Baldassarre2020GRAPHQAPM}, where the protein is in some
	way represented as a graph. This work brings the new idea of capturing the 3D structure of this graph 
	to improve the quality of GCN.



	\section{Problem statement}
	Consider a 3D model of a protein in space. The protein represents a chain of amino acids rolled up in space. 
	Through dividing space around a protein into cells, for example, by the Voronoi method, we can get a
	3D-graph, the vertexes of which are amino acids of protein and edges are carried out between those amino
	acids that are in adjacent cells. Denote the resulting graph by $G = (V, E)$, where verticles $V = (v_1, \ldots, v_n)$
	are a set of amino acids, $E$ are edges. For the $i$-th vertex we denote for $\mathcal{N}(v_i)$ the
	set of its neighbors in a graph $G$ and for $G$ an adjacency matrix $\boldsymbol{A}$:
	$$A_{ij} = \begin{cases}
		1, & (v_i, v_j) \in E \\
		0, & \text{otherwise}
	\end{cases}.$$ 
	Consider that each vertex $v_i$ is described by some real $d$-dimensional vector of attributes 
	$x(v_i) = \boldsymbol{x}_i$. In the simplest case, it can be a one-hot representation of an amino acid type.
	Using these data we will solve a regression problem: to predict for each vertex $v_i$ a real number - its
	"score". In other words, how correctly it is placed in the given 3D-model in comparison with the 
	actual conformation of this protein.
	%\begin{State}
	%    Мотивации и~интерпретации наиболее важны для понимания сути работы.
	%\end{State}
	
	%\begin{Theorem}
	%    Не~менее $90\%$ коллег, заинтересовавшихся Вашей статьёй,
	%    прочитают в~ней не~более~$10\%$ текста.
	%\end{Theorem}
	%
	%\begin{Proof}
	%    Причём это будут именно те~разделы, которые не содержат формул.
	%\end{Proof}
	%
	%\begin{Remark}
	%    Выше показано применение окружений
	%    Def, Theorem, State, Remark, Proof.
	%\end{Remark}
	
	
	%\section{Заключение}
	
	%Желательно, чтобы этот раздел был, причём он не~должен дословно повторять аннотацию.
	%Обычно здесь отмечают,
	%каких результатов удалось добиться,
	%какие проблемы остались открытыми.
	
	
	\bibliographystyle{plain}
	\bibliography{Pavlichenko2020Project52}
	
	% Решение Программного Комитета:
	%\ACCEPTNOTE
	%\AMENDNOTE
	%\REJECTNOTE
\end{document}
